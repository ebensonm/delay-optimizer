\documentclass[10pt]{elsarticle}

\usepackage{amsthm,amsmath,amsfonts,amssymb,amscd,mathrsfs}
\usepackage{txfonts}%,pxfonts}
\usepackage{supertabular,soul}
\usepackage[usenames,dvipsnames]{xcolor}
\usepackage{tikz, graphicx,color,geometry}
 \usepackage{multirow}
 \usetikzlibrary{arrows}
\usepackage[pdftex,
            pdfauthor={Reber},
            pdftitle={Time-Varying Time Delays and Intrinsic Stability},
            pdfsubject={Switched Systems, Network Theory, Nonlinear Dynamical Systems},
            pdfkeywords={Switched Systems, Network Theory, Nonlinear Dynamical Systems}]{hyperref}

\usepackage{bbm} %\mathbb numbers and other symbols
\usepackage{hyperref}
\usepackage{yfonts}
\usepackage{eucal}
\usepackage{overpic}
%\usepackage{breakurl}
\usetikzlibrary{calc}
\usepackage{enumitem}
\usepackage{dsfont}
\usepackage{commath}
\usepackage{xcolor}
\usepackage{soul}

\newcommand{\annotation}[1]{\marginpar{\tiny #1}}
\newcommand{\question}[1]{\medskip\noindent{\bf Question.} #1\medskip}
\newcommand\sHk[1]{{\bf\Large (CHECK: #1)}}
\newcommand{\comment}[1]{}

\DeclareMathOperator{\tr}{tr}
\def\big{\bigskip}
\def\m{\medskip}
\def\s{\smallskip}
\def\h{\hfill}
\def\dsp{\displaystyle}

\def \a{\alpha} \def \b{\beta} \def \g{\gamma} \def \d{\delta}
\def \t{\theta} \def \p{\phi} \def \e{\epsilon}
\def \l{\lambda} \def \z{\zeta} \def \o{\omega}

\newcommand{\defital}{\textit}
\newcommand{\ds}{\displaystyle}
\newcommand{\ZZ}{\mathbb Z}
\newcommand{\C}{\mathbb C}
\newcommand{\Q}{\mathbb Q}
\newcommand{\Aut}{\text{Aut}}
\newcommand{\diag}{\text{diag}}
\newcommand{\sL}{\mathscr{L}}
\newcommand{\cT}{\mathcal{T}}
\renewcommand{\so}{\mathscr{O}}
%Scripty Things
\renewcommand{\l}{\mathbf{\ell}}
\renewcommand{\r}{{\upsilon}}%%the inclusion of \IC into \cC
\renewcommand{\thefootnote}{\fnsymbol{footnote}}
%%%%%%%%%%%%%%%%%%%%%%%%%%%%%%%%%%%%%%%%%%%%%%%%%
\newtheorem{result}{Main Result}
\newtheorem{theorem}{Theorem}
\newtheorem{conjecture}{Conjecture}
\newtheorem{proposition}{Proposition}
\newtheorem{definition}{Definition}
\newtheorem{thm}{Theorem}[section]
\newtheorem{observ}[thm]{Observation}
\newtheorem*{thmstar}{Theorem.}
\newtheorem*{propstar}{Proposition}
\newtheorem{lemma}{Lemma}
\newtheorem*{main2}{Theorem A}
\newtheorem*{main3}{Theorem B}
\newtheorem{lem}[thm]{Lemma}
\newtheorem{sublm}[thm]{Sub-Lemma}
\newtheorem{prop}[thm]{Proposition}
\newtheorem{property}{\propertyautorefname}
%\renewcommand{\theproperty}{(\fnsymbol{property})}
\newtheorem{cor}[thm]{Corollary}
\newtheorem{conj}[thm]{Conjecture}
\newtheorem{quest}[thm]{Question}
\newtheorem{assumption}[thm]{Convention}
\newtheorem{remark}[thm]{Remark}
\newtheorem{rems}[thm]{Remarks}
\newtheorem{ax}[thm]{Axiom}
\newtheorem{example}[thm]{Example}
%\newtheorem*{example}[thm]{Example}
\newtheorem*{examplestar}{Example}
\theoremstyle{remark}
\newtheorem{notat}{Notation}

%\renewcommand{\thenota}{}
\DeclareMathOperator{\Sing}{Sing}
\DeclareMathOperator{\fix}{Fix}
\providecommand*{\propertyautorefname}{Property}

\setlength{\marginparwidth}{0.8in}
\let\oldmarginpar\marginpar
\renewcommand\marginpar[1]{\oldmarginpar[\raggedleft\footnotesize #1]%
{\raggedright\footnotesize #1}}

\begin{document}


\section*{Gradient Descent Time Delays Example}

\subsection*{Gradient Descent Algorithm}
\begin{equation}
x_k = x_{k-1} - \alpha\nabla(f(x_{k-1}))
\end{equation}

\subsection*{Gradient Descent With Delays}
Given delay matrix $D=[d_{ij}]\in\mathbb{N}^{n\times n}$.
\[
\begin{bmatrix}
   x_{k_{1}}\\
   x_{k_{2}}\\
   x_{k_{3}}\\
   \vdots\\
   x_{k_{n}}
\end{bmatrix} =
\begin{bmatrix}
   x_{k-1-d_{{11}_1}}\\
   x_{k-1-d_{{22}_2}}\\
   x_{k-1-d_{{33}_3}}\\
   \vdots\\
   x_{k-1-d_{{nn}_n}}\\
\end{bmatrix} - \alpha
\begin{bmatrix}
   \nabla(f(x_{k-1-d_{{11}_1}}, x_{k-1-d_{{12}_2}}, \dots, x_{k-1-d_{{1n}_n}}))_{1} \\
   \nabla(f(x_{k-1-d_{{21}_1}}, x_{k-1-d_{{22}_2}}, \dots, x_{k-1-d_{{2n}_n}}))_{2} \\
   \nabla(f(x_{k-1-d_{{31}_1}}, x_{k-1-d_{{32}_2}}, \dots, x_{k-1-d_{{3n}_n}}))_{3} \\
   \vdots\\
   \nabla(f(x_{k-1-d_{{n1}_1}}, x_{k-1-d_{{n2}_2}}, \dots, x_{k-1-d_{{nn}_n}}))_{n} \\
\end{bmatrix}
\]
\subsection*{Example of Non-Symmetric Delays}
Given Delay matrix
\begin{equation}
D = \begin{bmatrix}
1&2 \\
3&4 
\end{bmatrix}
\end{equation}
and function
\begin{equation}
f(x,y) = x^2y+x^2y^2
\end{equation}
\begin{equation}
\nabla(f(x,y)) = [2xy+2xy^2, x^2+2x^2y]
\end{equation}
We will also just define part of a time series so that we can actually see what the delays do.
Current time series, which is all the past states until now, is
\[
T_s = \begin{bmatrix}
3&1 \\
2&3 \\
4&5 \\
1&4 \\
3&6
\end{bmatrix}
\]

The first update using the delay matrix and time series, (we will select indeces with $1$ as the first row and column)
\[
\begin{bmatrix}
    x_{6_1}\\
    x_{6_2}
\end{bmatrix} = 
\begin{bmatrix}
    x_{{5-1}_{1}}\\
    x_{{5-4}_{2}}   
\end{bmatrix} - \alpha
\begin{bmatrix}
    \nabla(f(x_{{5-1}_{1}},x_{{5-2}_{2}}))_{1} \\
    \nabla(f(x_{{5-3}_{1}},x_{{5-4}_{2}}))_{2}
\end{bmatrix}
\]
\[
 = 
\begin{bmatrix}
   x_{{4}_{1}} \\
   x_{{1}_{2}}
\end{bmatrix} - \alpha
\begin{bmatrix}
    \nabla(f(x_{{4}_{1}},x_{{3}_{2}}))_{1} \\
    \nabla(f(x_{{2}_{1}},x_{{1}_{2}}))_{2}
\end{bmatrix} = 
\begin{bmatrix}
    1 \\
    1
\end{bmatrix} - \alpha
\begin{bmatrix}
    \nabla(f(1,5))_{1} \\
    \nabla(f(2,1))_{2}
\end{bmatrix}
\]
\[
 = 
\begin{bmatrix}
    1 \\
    1
\end{bmatrix} - \alpha
\begin{bmatrix}
    [60,11]_{1} \\ 
    [8, 12]_{2}
\end{bmatrix} = 
\begin{bmatrix}
    1 \\
    1
\end{bmatrix} - \alpha
\begin{bmatrix}
    60 \\
    12
\end{bmatrix}
\]
\subsection*{Example of Symmetric Delays}
Everything is the same as the previous example except 
\begin{equation}
D = \begin{bmatrix}
1&2 \\
1&2 
\end{bmatrix}
\end{equation}

The first update from the current time series
\[
\begin{bmatrix}
    x_{6_1}\\
    x_{6_2}
\end{bmatrix} = 
\begin{bmatrix}
    x_{{5-1}_{1}}\\
    x_{{5-2}_{2}}   
\end{bmatrix} - \alpha
\begin{bmatrix}
    \nabla(f(x_{{5-1}_{1}},x_{{5-2}_{2}}))_{1} \\
    \nabla(f(x_{{5-1}_{1}},x_{{5-2}_{2}}))_{2}
\end{bmatrix}
\]
\[
 = 
\begin{bmatrix}
   x_{{4}_{1}} \\
   x_{{3}_{2}}
\end{bmatrix} - \alpha
\nabla(f(x_{{4}_{1}},x_{{3}_{2}})) = 
\begin{bmatrix}
    1 \\
    5
\end{bmatrix} - \alpha
\nabla(f(1,5))
\]
\[
 = 
\begin{bmatrix}
    1 \\
    5
\end{bmatrix} - \alpha
\begin{bmatrix}
    60 \\
    11 
\end{bmatrix}
\]


\end{document}



































